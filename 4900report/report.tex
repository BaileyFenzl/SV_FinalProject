\documentclass{llncs}

\usepackage{listings}

\lstdefinelanguage{Coq}{
  ,morekeywords={match,end,Definition,Inductive,Lemma,Record,
    Variable,Section,case,of,if,then,else,is,let,in,do,return,with}%
  ,keywordstyle=\bfseries
  ,basicstyle=\sffamily
  ,columns=fullflexible
  ,numberstyle=\tiny
  ,escapeinside={@}{@}
  ,literate=
  {<-}{{$\leftarrow\;$}}1
  {=>}{{$\rightarrow\;$}}1
  {->}{{$\rightarrow\;$}}1
  {<->}{{$\leftrightarrow\;$}}1
  {<==}{{$\leq\;$}}1
  {\\/}{{$\vee\;$}}1
  {/\\}{{$\land\;$}}1
}
\lstset{language=Coq}

\begin{document}

\title{Huffman Coding}

\author{Bailey Fenzl}
\institute{Ohio University, Athens, OH 45701}

\maketitle

\section{Introduction}

For my final project, I choose to implement Huffman Coding in Coq. Huffman encoding is useful because of it's applications in compressing data 

\subsection{Subsection1} This is a subsection.

\subsection{Subsection2} This is a second subsection.

This is a citation.~\cite{gennaro2010non}

This is a chunk of code:
\begin{lstlisting}
  Definition f(x : nat) := S x.
  Definition g(y : nat) := f y.
\end{lstlisting}

This is inline code \lstinline|Fixpoint f(x : nat) := ...| typeset within a line of text.

\paragraph{Para1.} This is a paragraph, or subsubsection.

\section{Conclusion}

\bibliographystyle{plain}
\bibliography{references}

\end{document}
